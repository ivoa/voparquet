\documentclass[11pt,a4paper]{ivoa}
\input tthdefs
\input gitmeta

\title{Parquet in the VO}

\ivoagroup{Applications}

\author[https://wiki.ivoa.net/twiki/bin/view/IVOA/MarkTaylor]
       {Mark Taylor}
\author[https://wiki.ivoa.net/twiki/bin/view/IVOA/GregoryDuboisFelsmann]
       {Gregory Dubois-Felsmann}
\author[https://wiki.ivoa.net/twiki/bin/view/IVOA/FrancoisXavierPineau]
       {Fran\c{c}ois-Xavier Pineau}
\author{Brigitta Sip\H{o}cz}

\editor{Mark Taylor}

% \previousversion[????URL????]{????Concise Document Label????}
\previousversion{This is the first public release}


\begin{document}
\begin{abstract}
Parquet is an efficient file format for tabular data,
with widespread industry tool support.
It is being adopted by several astronomy projects for bulk storage and
distribution of large table products.
This Note discusses best practice for use of parquet within the VO,
in particular the use of VOTable to attach rich astronomical metadata
to otherwise metadata-poor parquet files.
\end{abstract}

\section*{Conformance-related definitions}

The words ``MUST'', ``SHALL'', ``SHOULD'', ``MAY'', ``RECOMMENDED'', and
``OPTIONAL'' (in upper or lower case) used in this document are to be
interpreted as described in IETF standard RFC2119 \citep{std:RFC2119}.

The \emph{Virtual Observatory (VO)} is a
general term for a collection of federated resources that can be used
to conduct astronomical research, education, and outreach.
The \href{https://www.ivoa.net}{International
Virtual Observatory Alliance (IVOA)} is a global
collaboration of separately funded projects to develop standards and
infrastructure that enable VO applications.


\section{Introduction}

Parquet is good for data but poor for metadata.
VOTable, the other way round.
Let's get them together and have some fun.

\appendix
\section{Changes from Previous Versions}

No previous versions yet.

% NOTE: IVOA recommendations must be cited from docrepo rather than ivoabib
% (REC entries there are for legacy documents only)
\bibliography{ivoatex/ivoabib,ivoatex/docrepo}

\end{document}
