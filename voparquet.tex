\documentclass[11pt,a4paper]{ivoa}
\input tthdefs
\input gitmeta

\title{Parquet in the VO}

\ivoagroup{Applications}

\author[https://wiki.ivoa.net/twiki/bin/view/IVOA/MarkTaylor]
       {Mark Taylor}
\author[https://wiki.ivoa.net/twiki/bin/view/IVOA/GregoryDuboisFelsmann]
       {Gregory Dubois-Felsmann}
\author[https://wiki.ivoa.net/twiki/bin/view/IVOA/FrancoisXavierPineau]
       {Fran\c{c}ois-Xavier Pineau}
\author{Brigitta Sip\H{o}cz}

\editor{Mark Taylor}

% \previousversion[????URL????]{????Concise Document Label????}
\previousversion{This is the first public release}


\begin{document}
\begin{abstract}
Parquet is an efficient file format for tabular data,
with widespread industry tool support.
It is being adopted by several astronomy projects for bulk storage and
distribution of large table products.
This Note discusses best practice for use of parquet within the VO,
in particular the use of VOTable to attach rich astronomical metadata
to otherwise metadata-poor parquet files.
\end{abstract}

\section*{Conformance-related definitions}

The words ``MUST'', ``SHALL'', ``SHOULD'', ``MAY'', ``RECOMMENDED'', and
``OPTIONAL'' (in upper or lower case) used in this document are to be
interpreted as described in IETF standard RFC2119 \citep{std:RFC2119}.

The \emph{Virtual Observatory (VO)} is a
general term for a collection of federated resources that can be used
to conduct astronomical research, education, and outreach.
The \href{https://www.ivoa.net}{International
Virtual Observatory Alliance (IVOA)} is a global
collaboration of separately funded projects to develop standards and
infrastructure that enable VO applications.


\section{Introduction}

The \href{https://parquet.apache.org/docs/}{Apache Parquet file format}
is an open column-oriented data storage
format first developed in 2013.
It offers per-column compression, dictionary encoding, and
a kind of column value indexing.
A number of data processing environments optimised for parallel
computing make use of these features to enable fast processing
of large or very large tables.
In 2024, a number of astronomy projects including Rubin, Gaia and SPHEREx
are using or planning to use parquet for storage, processing
and distribution of large-scale astronomical data products.
I/O libraries are available in many languages including Python,
Java, C++ and Rust, and these have been leveraged by astronomer-facing
software such as Astropy, CDS services and TOPCAT to facilitate
use of parquet data in astronomy.

While offering efficient data processing facilities however,
the standard metadata provided by Parquet files is quite primitive.
Apart from a name and datatype for each column,
there is only a list of untyped key-value pairs per table
and per column, with no standard semantics for the keys.
For scientific usability, more metadata is desirable and even necessary,
especially in view of the complexity of the data represented;
tables can easily contain hundreds of columns.
A minimum requirement is column attributes such as units, descriptions
and UCDs; in many cases additional information relating to
coordinate systems, service descriptors or processing flags
may also be required.

The VOTable format has been developed within the VO since its
inception to hold exactly the kind of metadata required here.
Combining the virtues of VOTable and Parquet therefore
can supply a format which delivers efficiency alongside
rich astronomical metadata.

This Note addresses the question of how to effect that combination.
In particular it provides a prescription for embedding
VOTable metadata in parquet files in a way which will be interoperable
between data producers and consumers from different projects.
Although usage may be refined in future as the result of
developing requirements and implementation experience,
the intention (at least, the hope) is that the
prescriptions here will remain valid as a backwardly compatible
baseline for some while, so that future iterations of parquet I/O
software in the VO will remain compatible with files written
according to this Note.

\subsection{Scope}

The topic of this Note invites other discussions, including
best practice for sharding large datasets among multiple parquet files,
policy for choice of compression algorithms within parquet,
and the application of similar ideas to enhance other metadata-poor
file formats using VOTable.

This Note avoids those questions in the interest of achieving rapid
consensus on the question of combining VOTable and parquet.
Several projects will be generating large parquet collections
in the near future, so that early agreement on the basics of the format
is required to achieve interoperability between a number of
datasets too large to be rewritten at a later date.

Future work may build on the current document and on implementation
experience to produce a revised Note or a Recommendation-track document
that enhances the current proposal or addresses
some of these wider questions.



\appendix
\section{Changes from Previous Versions}

No previous versions yet.

% NOTE: IVOA recommendations must be cited from docrepo rather than ivoabib
% (REC entries there are for legacy documents only)
\bibliography{ivoatex/ivoabib,ivoatex/docrepo}

\end{document}
