\documentclass[11pt,a4paper]{ivoa}
\input tthdefs
\input gitmeta

\title{Parquet in the VO}

\ivoagroup{Applications}

\author[https://wiki.ivoa.net/twiki/bin/view/IVOA/MarkTaylor]
       {Mark Taylor}
\author[https://wiki.ivoa.net/twiki/bin/view/IVOA/GregoryDuboisFelsmann]
       {Gregory Dubois-Felsmann}
\author[https://wiki.ivoa.net/twiki/bin/view/IVOA/FrancoisXavierPineau]
       {Fran\c{c}ois-Xavier Pineau}
\author{Brigitta Sip\H{o}cz}

\editor{Mark Taylor}

% \previousversion[????URL????]{????Concise Document Label????}
\previousversion{This is the first public release}

\lstloadlanguages{XML}
\lstset{basicstyle=\ttfamily\scriptsize}

\newcommand{\voparquet}{VOParquet}


\begin{document}
\begin{abstract}
Parquet is a file format for record-based data,
with widespread industry tool support.
It is being adopted by several astronomy projects for bulk storage and
distribution of large tabular data products.
This Note discusses best practice for use of parquet within the VO,
and in particular defines the \voparquet\ convention
which uses VOTable to attach rich astronomical metadata
to otherwise metadata-poor parquet files.
\end{abstract}

\section*{Conformance-related definitions}

The words ``MUST'', ``SHALL'', ``SHOULD'', ``MAY'', ``RECOMMENDED'', and
``OPTIONAL'' (in upper or lower case) used in this document are to be
interpreted as described in IETF standard RFC2119 \citep{std:RFC2119}.

The \emph{Virtual Observatory (VO)} is a
general term for a collection of federated resources that can be used
to conduct astronomical research, education, and outreach.
The \href{https://www.ivoa.net}{International
Virtual Observatory Alliance (IVOA)} is a global
collaboration of separately funded projects to develop standards and
infrastructure that enable VO applications.


\section{Introduction}
\label{sec:intro}

The Apache Parquet file
format\footnote{\url{https://parquet.apache.org/docs/}}
is a column-oriented storage format for record-based data
first developed in 2013.
It offers per-column compression, dictionary encoding, and
a kind of column value indexing.
A number of data processing environments optimised for parallel
computing make use of these features to enable fast processing
of large or very large tables.
At the time of writing (2024),
several astronomy projects including Rubin, Gaia and SPHEREx
are using or planning to use parquet for storage, processing
and distribution of large-scale astronomical data products.
I/O libraries are available in various languages including Python,
Java, C++ and Rust, and these have been leveraged by astronomer-facing
software such as Astropy, CDS services and TOPCAT to facilitate
use of parquet data in astronomy.

While offering efficient data storage however,
the standard metadata provided by Parquet files is quite rudimentary.
Apart from a name and datatype for each column,
there is only a list of untyped key-value pairs per table
and per column-chunk, with no standard semantics for the keys.
For scientific usability, better metadata is desirable and even necessary,
especially in view of the complexity of the data represented;
astronomy tables can easily contain hundreds of columns.
A minimum requirement is column attributes such as units, descriptions,
UCDs and precisions; in many cases additional information relating to
coordinate systems, service descriptors or processing flags
may also be required.

The VOTable format \citep{2019ivoa.spec.1021O}
has been developed within the VO since its
inception to hold exactly the kind of metadata required here.
Combining the virtues of VOTable and Parquet therefore
can supply a format which delivers storage efficiency alongside
rich astronomical metadata.

This Note addresses the question of how to effect that combination.
In particular it defines a convention named {\voparquet}
that stores VOTable metadata in parquet files in a way
which will be interoperable
between data producers and consumers from different projects.
Although usage may be refined in future as the result of
developing requirements and implementation experience,
the intention (at least, the hope) is that the
prescriptions here will remain valid as a backwardly compatible
baseline for some while, so that future iterations of parquet I/O
software in the VO will remain compatible with files written
according to this Note.
For this reason, we restrict ourselves here to the minimum rules
that will enable interoperability, and avoid imposing requirements
on points of detail for which the best way forward is not obvious.

The normative part of this \voparquet\ convention
is quite short and can be found in Section~\ref{sec:serialization}
and especially Subsection~\ref{sec:format}.
The other sections provide context and discussion.

\subsection{Scope}
\label{sec:scope}

The topic of this Note suggests other discussions, including
best practice for sharding large datasets among multiple parquet files,
policy for choice of compression algorithms within parquet,
suitability of parquet for archival storage,
and the application of similar ideas to enhance other metadata-poor
file formats using VOTable.

This Note avoids those questions in the interest of achieving rapid
consensus on the question of combining VOTable and parquet.
Several projects will be generating large parquet collections
in the near future, so that early agreement on the basics of the format
is required to achieve interoperability between a number of
datasets too large to be rewritten at a later date.

Future work may build on the current document and on implementation
experience to produce a revised Note or a Recommendation-track document
that extends the current proposal or addresses
some of these wider questions.

\section{\voparquet\ Convention}
\label{sec:serialization}

\subsection{Serialization Approach}

The parquet and VOTable file formats both provide serialization
of tabular data, along with some degree of file- and column-level metadata.
Given an abstract input table with rich metadata,
the basic prescription for writing a \voparquet\ file is:
\begin{enumerate}
\item serialize the input table to VOTable but without including the data part,
      thus producing an XML document containing table metadata only
\item serialize the input table to parquet in the usual way, but
\item include the data-less VOTable document in the file-level metadata
      of the parquet file
\end{enumerate}
The parquet file generated in the final step is the \voparquet\ output.
When reading such a file:
\begin{enumerate}
\item read the parquet data in the usual way
\item search for a VOTable in the file-level metadata
\item if one is present, parse it and use the table- and column-level
      metadata it contains to decorate the data read from parquet
\end{enumerate}

The serialised table is therefore a perfectly legal parquet file,
which can be read by any parquet I/O software.
But \voparquet-aware software can use the attached dataless VOTable to recover
the rich metadata associated with the original table.

\subsection{Serialization Format}
\label{sec:format}

The VOTable metadata document stored in the parquet metadata
must contain a \xmlel{TABLE} element describing the parquet data table.
This looks exactly like a normal \xmlel{TABLE} element
except that it has no \xmlel{DATA} child;
such dataless tables are permitted by the VOTable schema.
In particular it must contain \xmlel{FIELD} elements describing the
columns of the parquet data,
and it may contain other elements such as \xmlel{PARAM}, \xmlel{COOSYS} etc
providing additional table-level metadata.
 
This \xmlel{DATA}-less \xmlel{TABLE} must be the first
\xmlel{TABLE} element in the VOTable document.
Other \xmlel{TABLE}s,
for instance providing auxiliary data or metadata,
may appear in the VOTable document,
but are not used to describe the parquet data directly.
The VOTable document must be a schema-valid and legal VOTable instance.
No particular VOTable version is mandated by this convention.

An example VOTable metadata document describing a 3-column table
might look like this\footnote{
  The apparent mismatch between version and namespace is intended;
  see VOTable 1.4 section 3 for details.}:
\lstinputlisting{metadata-example.vot}

This dataless VOTable document is stored in the
{\tt key\_value\_metadata} list of the
{\tt FileMetaData} structure in the parquet footer.
That list is defined by the parquet file
format\footnote{\url{https://github.com/apache/parquet-format}}
to contain an unstructured collection of string-string key-value pairs,
and is available for applications to populate with arbitrary metadata.
The \voparquet\ convention requires the following key-value pairs
to be present:
\begin{bigdescription}
\item[{\tt IVOA.VOTable-Parquet.version}]
   The version of this convention.  Must be "1.0" at this version.
\item[{\tt IVOA.VOTable-Parquet.content}]
   The content of the data-less VOTable document described above,
   encoded using UTF-8.
   An XML declaration (``\verb|<?xml ... ?>|'') may optionally
   precede the content,
   but if present it must not declare a non-UTF-8 encoding.
\end{bigdescription}

\subsection{Data/Metadata Mismatches}

In the most straightforward case, there will be a one-to-one mapping
of columns in the parquet data table and the VOTable metadata table,
where the $N^{\rm th}$ \xmlel{FIELD} of the VOTable metadata table
provides an accurate description of the $N^{\rm th}$ top-level column
($N^{\rm th}$ {\tt SchemaElement} child of the root {\tt SchemaElement})
of the parquet data table,
in particular declaring VOTable \xmlel{datatype} and \xmlel{arraysize}
attributes that
correspond to the physical/logical type of that parquet column.
It is expected that for many or most tables in the VO this will be the case.

However, the data models of VOTable and Parquet do not exactly match,
and it is possible to store columns in a parquet file that cannot
accurately be described by VOTable metadata.
Columns containing scalars and 1-d arrays of booleans,
unsigned 8-bit integers, signed 16/32/64-bit integers
and strings map straightforwardly to VOTable,
but other types such as unsigned 64-bit integers
and structured data do not.

In the case of such mismatches,
it is recommended that for each top-level parquet column,
\voparquet\ writers should write exactly one VOTable \xmlel{FIELD}
element that makes a best effort to describe the data.
If a parquet column has a type that does not resemble any VOTable datatype,
for instance structured data, it is recommended to describe it as a
VOTable string, e.g.\ \verb|datatype="char"| \verb|arraysize="*"|.
It is not permitted to omit the \xmlel{datatype} attribute altogether,
since that would result in an illegal VOTable document.
Other approaches are possible, for instance describing
the primitive sub-elements of a structured parquet top-level column
using multiple VOTable \xmlel{FIELD}s,
but these are not currently recommended.

Readers must treat the parquet data and datatypes as authoritative,
and may make use of as much VOTable column metadata as is feasible,
for instance using units and UCDs where present but discarding
VOTable datatypes that are incompatible with the parquet data columns.
Ignoring the metadata table \xmlel{datatype} and \xmlel{arraysize} attributes
on \xmlel{FIELD} elements altogether in favour of column types acquired
from the parquet parsing would be one way for readers to handle this.
If the number of VOTable columns cannot be reconciled with the parquet data,
the reader must discard the VOTable column metadata.
The reader is free to discard some or all of the VOTable metadata
in case of any other difficulty in reconciling the VOTable metadata
and parquet data, or for any other reason.

Future evolutions of this convention may refine or adjust this advice.

\section{Alternative Approaches}

Other approaches have been suggested for association of rich metadata
with parquet files.
These include variations on the idea of
using a VOTable document stored {\em outside\/} of the described parquet
serialization,
and use of the existing per-table and per-column-chunk key-value lists
defined by the parquet format for direct (non-VOTable) storage
of metadata items.

This Note neither requires nor forbids use of these alternative
approaches either alongside or instead of the \voparquet\ convention
described here, nor does it prescribe how to reconcile
the redundancy and possible conflicts that may arise from
storage of metadata in more than one place.
It does encourage use of the \voparquet\ format where interoperability
within the VO is a goal,
but parquet producers and consumers may wish to consider these other
options as well.



\appendix
\section{Changes from Previous Versions}

No previous versions yet.

% NOTE: IVOA recommendations must be cited from docrepo rather than ivoabib
% (REC entries there are for legacy documents only)
\bibliography{ivoatex/ivoabib,ivoatex/docrepo}

\end{document}
